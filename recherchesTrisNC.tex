\documentclass[french]{article}
\usepackage[T1]{fontenc}
\usepackage[utf8]{inputenc}
\usepackage{lmodern}

\usepackage[a4paper,left=3cm,right=3cm,top=2.5cm,bottom=2.5cm]{geometry}
\usepackage{babel}

\usepackage{fancyhdr}
\pagestyle{fancy}
\renewcommand{\headrulewidth}{1pt}
\fancyhead[L]{Algorithmique et Programmation Avancée}
\fancyfoot[R]{Ebersold, Thieblin, Desprat}
\fancyhead[R]{Département Mathématiques-Informatique}
\fancyfoot[L]{MIA0301V - MI00304}


\usepackage{listingsutf8}
\usepackage{listings}
\usepackage{xcolor}
\lstset { %
	language=C++,
    basicstyle=\footnotesize\ttfamily,
    keywordstyle=\color{blue}\ttfamily,
    stringstyle=\color{red}\ttfamily,
    commentstyle=\color{olive}\ttfamily,
    morecomment=[l][\color{magenta}]{\#},
	backgroundcolor=\color{black!5}, % set backgroundcolor
	numbers=left, 
    numberstyle=\tiny\ttfamily, 
    breaklines=true,
    numbersep=5pt,
    xleftmargin=.25in,
    xrightmargin=.25in
}
\lstset{%
	inputencoding=utf8,
	extendedchars=true,
	literate=%
	{é}{{\'e}}{1}%
	{è}{{\`e}}{1}%
	{à}{{\`a}}{1}%
	{ç}{{\c{c}}}{1}%
	{œ}{{\oe}}{1}%
	{ù}{{\`u}}{1}%
	{É}{{\'E}}{1}%
	{È}{{\`E}}{1}%
	{À}{{\`A}}{1}%
	{Ç}{{\c{C}}}{1}%
	{Œ}{{\OE}}{1}%
	{Ê}{{\^E}}{1}%
	{ê}{{\^e}}{1}%
	{î}{{\^i}}{1}%
	{ô}{{\^o}}{1}%
	{û}{{\^u}}{1}%
	{ë}{{\¨{e}}}1
	{û}{{\^{u}}}1
	{â}{{\^{a}}}1
	{Â}{{\^{A}}}1
	{Î}{{\^{I}}}1
}


\begin{document}
	
	\begin{minipage}{\textwidth}
\begin{center}

{\Large Mise en œuvre avec C++ \\ Feuille d'exercices : \textbf{Recherches et tris}}
\end{center}
	\end{minipage}

\section{Recherche d'un élément dans un tableau}

	Au sein d’un programme \texttt{recherche\_1.cpp}, écrire un sous-programme qui prend en paramètres un tableau non trié et une valeur à rechercher, et qui retourne l’indice de l’élément recherché à l’intérieur du tableau s’il existe, -1 sinon.
\section{Recherche d'un élément dans un tableau trié}
	Même question pour un \textbf{tableau trié} \texttt{recherche\_2.cpp}.
	\section{Recherche d'un élément dans un tableau trié par dichotomie}
	Même question pour un tableau trié, mais la recherche doit se faire par \textbf{dichotomie} \texttt{recherche\_3.cpp}
\section{Tri par insertion}
Au sein d’un programme \texttt{tri\_insertion.cpp}, écrire un sous-programme qui prend en paramètre un tableau quelconque et le retourne trié. Le tri se fait selon la méthode de l’insertion. Le principe de ce tri est très simple : c'est le tri que toute personne utilise quand elle a des dossiers (ou n'importe quoi d'autre) à classer. On prend un dossier et on le met à sa place parmi les dossiers déjà triés. Puis on recommence avec le dossier suivant. Pour procéder à un tri par insertion, il suffit de parcourir un tableau : on prend les éléments dans l'ordre. Ensuite, on les compare avec les éléments précédents jusqu'à trouver la place de l'élément qu'on considère. Il ne reste plus qu'à décaler les éléments du tableau pour insérer l'élément considéré à sa place dans la partie déjà  triée.
\section{Tri par sélection}
Même problème en utilisant un tri par sélection. Le tri par sélection consiste en la recherche soit du plus grand élément (ou le plus petit) que l'on va replacer à sa position finale c'est-à-dire en dernière position (ou en première), puis on recherche le second plus grand élément (ou le second plus petit) que l'on va replacer également à sa position finale c'est-à-dire en avant- dernière position (ou en seconde), etc., jusqu'à ce que le tableau soit entièrement trié.
\section{Tri à bulles}
Même problème en utilisant un tri à bulles. Le tri à bulles consiste à faire remonter le plus grand élément du tableau (comme une bulle d'air remonte à la surface) en comparant les éléments successifs. C'est-à-dire qu'on va comparer le 1er et le 2e élément du tableau, conserver le plus grand et puis les échanger s'ils sont désordonnés les uns par rapport aux autres. On recommence cette opération jusqu'à la fin du tableau. Ensuite, il ne reste plus qu'à renouveler cela jusqu'à l'avant-dernière place et ainsi de suite..

\end{document}