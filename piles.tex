\documentclass[french]{article}
\usepackage[T1]{fontenc}
\usepackage[utf8]{inputenc}
\usepackage{lmodern}

\usepackage[a4paper,left=3cm,right=3cm,top=2.5cm,bottom=2.5cm]{geometry}
\usepackage{babel}

\usepackage{fancyhdr}
\pagestyle{fancy}
\renewcommand{\headrulewidth}{1pt}
\fancyhead[L]{Algorithmique et Programmation Avancée}
\fancyfoot[R]{Ebersold, Thieblin, Desprat}
\fancyhead[R]{Département Mathématiques-Informatique}
\fancyfoot[L]{MIA0301V - MI00304}


\usepackage{listingsutf8}
\usepackage{listings}
\usepackage{xcolor}
\lstset { %
	language=C++,
    basicstyle=\footnotesize\ttfamily,
    keywordstyle=\color{blue}\ttfamily,
    stringstyle=\color{red}\ttfamily,
    commentstyle=\color{olive}\ttfamily,
    morecomment=[l][\color{magenta}]{\#},
	backgroundcolor=\color{black!5}, % set backgroundcolor
	numbers=left, 
    numberstyle=\tiny\ttfamily, 
    breaklines=true,
    numbersep=5pt,
    xleftmargin=.25in,
    xrightmargin=.25in
}
\lstset{%
	inputencoding=utf8,
	extendedchars=true,
	literate=%
	{é}{{\'e}}{1}%
	{è}{{\`e}}{1}%
	{à}{{\`a}}{1}%
	{ç}{{\c{c}}}{1}%
	{œ}{{\oe}}{1}%
	{ù}{{\`u}}{1}%
	{É}{{\'E}}{1}%
	{È}{{\`E}}{1}%
	{À}{{\`A}}{1}%
	{Ç}{{\c{C}}}{1}%
	{Œ}{{\OE}}{1}%
	{Ê}{{\^E}}{1}%
	{ê}{{\^e}}{1}%
	{î}{{\^i}}{1}%
	{ô}{{\^o}}{1}%
	{û}{{\^u}}{1}%
	{ë}{{\¨{e}}}1
	{û}{{\^{u}}}1
	{â}{{\^{a}}}1
	{Â}{{\^{A}}}1
	{Î}{{\^{I}}}1
}


\begin{document}
	
	\begin{minipage}{\textwidth}
		\begin{center}
			
			{\Large Algorithmique et Programmation Avancée - Mise en œuvre avec C++ \\ Feuille d'exercices : \textbf{les Piles}}
		\end{center}
	\end{minipage}
	\section{Implantez les opérations d’une pile spécifiées dans le cours}
On désire réaliser la notion de pile à l'aide d'une structure de données. 
Le type \texttt{Pile} est une structure de données comportant 2 \textbf{champs} :
\begin{description}
	\item[Elts] : tableau devant contenir les éléments de la pile. La capacité de la pile est définie par~\texttt{MAX}\footnote{En C++ pour définir une macro \texttt{NOMMACRO} associée à une valeur \texttt{MAVALEUR}, on utilise \texttt{\#define NOMMACRO MAVALEUR} -- i.e. à chaque fois qu'on utilise \texttt{NOMMACRO}, la valeur \texttt{MAVALEUR} associée sera utilisée. Ici, on s'en sert pour définir une constante \texttt{MAX} à 100}. Les éléments de ce tableau sont de type \textbf{entiers}.
	\item[NbElts] : nombre d’éléments dans la pile. On supposera que les éléments de la pile sont rangés dans l’ordre des indices croissants. Pour une pile vide, NbElts doit être égal à zéro.
\end{description}

\textbf{Trouver les erreurs} le Listing \ref{list1} et \textbf{écrire} le code corrigé.
	\begin{lstlisting}[caption={A corriger: Structure Pile},label=list1]
#define MAX 100
Struc Pile{
	tableauChar Elts[MAX];
	entier NbElts
} 
\end{lstlisting}

\section{Ecrire un sous-programme vérifiant l’existence d’un élément de valeur << e >> dans la pile}
Ce traitement devra être basé sur une \textbf{recherche dichotomique}.

\section{Utilisation d’une pile : Application}
\textbf{Ecrire} un algorithme qui vérifie qu’un texte contenant des caractères standards est syntaxiquement correct du point de vue des parenthèses.

Les parenthèses sont de trois types : \texttt{ (, \{, [ } et leurs parenthèses fermantes correspondantes sont respectivement \texttt{), \}, ]}.
$\rightarrow$ on peut commencer juste par les parenthèses
\begin{lstlisting}[caption={Rappel : Déclaration d'une chaîne de caractères},label=stringrappel]
char x[] = "une chaine de caractères avec des parenthèses ())())))))(((())";

cout << x[0] << endl;//la console affiche : u
\end{lstlisting}

La correction syntaxique implique qu’à chaque parenthèse ouvrante corresponde une parenthèse fermante du même type, plus loin dans le texte.


Le texte compris entre ces deux parenthèses devra également être correct du point de vue des parenthèses.
\end{document}