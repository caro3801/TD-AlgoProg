\documentclass[french]{article}
\usepackage[T1]{fontenc}
\usepackage[utf8]{inputenc}
\usepackage{lmodern}

\usepackage[a4paper,left=3cm,right=3cm,top=2.5cm,bottom=2.5cm]{geometry}
\usepackage{babel}

\usepackage[colorinlistoftodos]{todonotes}
\newcommand{\elodie}[2][]{\todo[bordercolor=teal,color=teal!25, #1]{Elodie: {\small #2}}}
\newcommand{\caroline}[2][]{\todo[bordercolor=yellow,color=yellow!25, #1]{Caroline: {\small #2}}}
\newcommand{\sophie}[2][]{\todo[bordercolor=red,color=red!25, #1]{Sophie: {\small #2}}}
\newcommand{\R}[0]{\mathbb{R}}

\usepackage{fancyhdr}
\pagestyle{fancy}
\renewcommand{\headrulewidth}{1pt}
\fancyhead[L]{Algorithmique et Programmation Avancée}
\fancyfoot[R]{Ebersold, Thieblin, Desprat}
\fancyhead[R]{Département Mathématiques-Informatique}
\fancyfoot[L]{MIA0301V - MI00304}


\usepackage{listingsutf8}
\usepackage{listings}
\usepackage{xcolor}
\lstset { %
	language=C++,
    basicstyle=\footnotesize\ttfamily,
    keywordstyle=\color{blue}\ttfamily,
    stringstyle=\color{red}\ttfamily,
    commentstyle=\color{olive}\ttfamily,
    morecomment=[l][\color{magenta}]{\#},
	backgroundcolor=\color{black!5}, % set backgroundcolor
	numbers=left, 
    numberstyle=\tiny\ttfamily, 
    breaklines=true,
    numbersep=5pt,
    xleftmargin=.25in,
    xrightmargin=.25in
}
\lstset{%
	inputencoding=utf8,
	extendedchars=true,
	literate=%
	{é}{{\'e}}{1}%
	{è}{{\`e}}{1}%
	{à}{{\`a}}{1}%
	{ç}{{\c{c}}}{1}%
	{œ}{{\oe}}{1}%
	{ù}{{\`u}}{1}%
	{É}{{\'E}}{1}%
	{È}{{\`E}}{1}%
	{À}{{\`A}}{1}%
	{Ç}{{\c{C}}}{1}%
	{Œ}{{\OE}}{1}%
	{Ê}{{\^E}}{1}%
	{ê}{{\^e}}{1}%
	{î}{{\^i}}{1}%
	{ô}{{\^o}}{1}%
	{û}{{\^u}}{1}%
	{ë}{{\¨{e}}}1
	{û}{{\^{u}}}1
	{â}{{\^{a}}}1
	{Â}{{\^{A}}}1
	{Î}{{\^{I}}}1
}


\begin{document}
	
	\begin{minipage}{\textwidth}
\begin{center}

{\Large Mise en œuvre avec C++ \\ Feuille d'exercices : \textbf{les Listes}}
\end{center}
	\end{minipage}
	
	\section*{Notions d'implantation à connaître}
	En C++, Il est possible de créer ses propres types, à partir des types existants.
	Ainsi, les structures de données se définissent de la façon suivante :
	Exemple : le type date
	
	\begin{lstlisting}[caption={le type \textbf{date}}]
struct date {
	int jour ; 
	int mois ; 
	int annee ;
};
	\end{lstlisting}
	
	Chaque \textbf{variable} de type \texttt{date} est donc composée de trois parties appelées ses « champs » (\texttt{jour,mois,annee}). Chaque \textbf{champ} est caractérisé par son \textbf{type} et son \textbf{identificateur}.
	
	Accès aux champs d'une variable composée : utiliser l’opérateur << \textbf{.} >> suivi de l’identificateur du champ. Ainsi : \texttt{id\_variable.id\_champ} permet d’accéder au champ \texttt{id\_champ} de la variable \texttt{id\_variable}.
	\begin{lstlisting}[caption={Utilisation du type date}]
//utilisation comme argument dans une fonction (qui ne modifie pas les valeurs de la structure)
void afficherDate(date d){
    cout << d.jour << "/" << d.mois << "/" << d.annee << endl;
}
//utilisation comme argument dans une fonction qui modifie les valeurs de la structure
void modifierMois(date d, int nouveauMois){
    d.mois = nouveauMois;
}
//appel des fonctions depuis le main
int main() {
    	date d; // declaration d'une variable de type date
    	d.jour = 7; //affectation des champs de la variable d
    	d.mois = 6;
    	d.annee = 2012;
    	modifierMois(d, 8); 
    	afficherDate(d);s
    	system("PAUSE");
}
	\end{lstlisting}
	
	\section{Ecrire le type ListeIterative en C++}
	\subsection{Donnez la structure d’une liste itérative implantée par un tableau}
	Considérons les sous-programmes suivants dans une liste d’entiers (\texttt{int tab[10]}):
\begin{lstlisting}[caption={Liste des opérations dans une liste d’entiers}]
// creation d'une liste vide 
Liste listeVide()
// le rajout d un élément se fait en tête de liste
void insererAuDebut(Liste &l, int element){} 
// On peut vider la liste dans sa totalite
void purger(Liste &l){} 
//On supprime l élément repere par sa valeur
boolean effacerElement(Liste &l, int element){}
// teste si la liste est vide
boolean estVide(Liste &l){} 
// retourne la longueur de la liste
int longueur(Liste &l){}
//retourne la valeur de l élément en tete de liste
int tete(Liste &l){}
// recherche si l entier element est present dans la liste 
// si oui le programme retourne son index 
// sinon s il n est pas dans la liste le programme retourne -1.
int rechercher (Liste &l, int element){}  
\end{lstlisting}
	
	
	\subsection{A partir de ces spécifications, donnez le code de ces opérations}

	\section{Constitution d’une liste}
	Construire la liste d’entier L=<6, 55, 444>.
	
	\begin{lstlisting}
int main(){
    Liste l= listeVide(); //creation liste vide
    cout << "Insertion de 444" << endl;
    insererAuDebut(l,444);
    afficherListe(l);
    cout << "Insertion de 55" << endl;
    insererAuDebut(l,55);
    afficherListe(l);
    cout << "Insertion de 6" << endl;
    insererAuDebut(l,6);
    afficherListe(l);
	
    //Exemple d'utilisation de la fonction effacerElement
    cout << "Suppression de 55" << endl;
    effacerElement(l,55);
    afficherListe(l);
    
    cout << "Purge" << endl;
    purger(l);
    afficherListe(l);
    cin.get(); //attendre que l'utilisateur appuie sur ENTREE
    return 0;	 
}
	\end{lstlisting}
	\section{Ecrire le type ListeIterative en C++ avec insertion, suppression à un index donné}
	A partir des opérations décrites en cours, donnez le code d’une ListeIterative
	
	\begin{lstlisting}
struct Liste{
    int T[10];//tableau d'entiers (ici taille max = 10)
    int NbEl;
};//declaration d'une structure finie par un point-virgule

//fonction qui cree une liste vide 
Liste listeVide(){
    //creation de la liste
    //preciser que la liste est vide
    //renvoyer la liste
}

//Fonction qui renvoie le premier element de la liste
int tete(Liste &l){}

//renvoie -1 si l'element n'est pas dans la liste, son indice sinon
int rechercher (Liste l, int e){
    //tant que je n'ai pas trouve e et que je suis encore dans le tableau
        //mon element est l'element recherche?
 	   	//OUI
 	   	    //recuperer l'index
 	   	//NON
 	   	    //je vais regarder l'element suivant
    //retourner la valeur de l'index
}

//Fonction pour savoir si une liste est vide
bool estVide(Liste &l){
    // Si l.NbEl==0, renvoie True,  false sinon
}

//fonction qui renvoie la longueur d'une liste
int longueur(Liste &l){}

//Fonction qui insere un element a l'indice index du tableau
void inserer(Liste &l, int index, int e){
    //verification qu'il reste de la place pour un element
    //sinon
        // parcours du tableau dans le sens inverse 
        // décalage de tous les éléments à droite de l'indice d'insertion vers la droite
        
        //affectation de la tete de liste avec le nouvel element
        //incrementation du nombre d'elements dans la liste
    }
}

//Fonction qui retire l'element a l'indice index de la liste
void supprimer(Liste &l, int index){
    //verifier que l'indice est dans le tableau
        //parcours de la liste de l'element a enlever a l'avant dernier element
        //decalage vers la gauche des elements apres celui a enlever
          
        //decrementation du nb d'elements
    }
}

//fonction pour afficher la liste
void afficherListe(Liste &l){
    cout << "La liste contient : " ;
    for (int i=0; i <l.NbEl; i++){
        cout << l.T[i] << " ";
    }
    cout <<endl;
}
	\end{lstlisting}
	
	\section{Constitution d’une liste}
	Construire alors la liste d’entier L=<6, 55, 444>.
	
	\begin{lstlisting}
int main(){
     //creation liste vide
    cout << "Insertion de 444 a la position 0" << endl;
    //appeler le sous programme insertion avec les bons paramètres
    //afficher la liste
    cout << "Insertion de 6 a la position 0" << endl;
    //appeler le sous programme insertion avec les bons paramètres
    //afficher la liste
    cout << "Insertion de 55 a la position 1" << endl;
    //appeler le sous programme insertion avec les bons paramètres
    //afficher la liste
    cout << "Suppression de l'element a la position 1" << endl;
    //appeler le sous programme supprimer avec les bons paramètres
    //afficher la liste
    cin.get();//attendre que l'utilisateur appuie sur ENTREE
    return 0;	 
}
\end{lstlisting}
	
	\section{Améliorations, contrôles}
	Pour toutes ces opérations, proposez des contrôles qui permettraient de prendre en compte et d’éviter les cas d’erreurs ....
\end{document}
